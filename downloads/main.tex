\documentclass{article}
\usepackage[utf8]{inputenc}

\title{The speed of sound in air}
\author{tridao0104 }
\date{March 2016}

\begin{document}

\maketitle

\section{Introduction}
The	purpose	of	this	lab	is	to	measure	the	speed	of	sound	in	air	by	determining	
the	wavelength	of	resonant	sound	waves	in	an	air	column	of	variable	length. In	
this	lab,	sound	waves	will	be	generated	by	an Apple smart phone.	The	medium	is	
room	temperature	air	enclosed	in	a	glass	tube,	sealed	at	the	lower	end	by	a	
column	of	water.	When	we turn on the phone,	a	train	of	waves	
consisting	of	alternate	compressions	and	rarefactions	in	air	is	sent	down	the	
tube.	This	wave	train	is	reflected	at	the	water	surface	with	a	phase	change	of	180	
degrees and	passes	back	up	the	tube.	At	the	open	end	of	the	tube,	it	is	again	
reflected,	but	with	no	phase	change in	this	case.	We	have	to	determine	several	
effective	lengths	of	the	tube	at	which	the	resonance	occurs	for	each	vibration.
Another	objective	of	this	lab	was	to	determine	the	wave	length	of	the	wave	for	
each	vibration	from	the	effective	length	of	the	resonance	tube. Lastly,	we	will	
compare	the	measured	speed	of	sound	with	the	accepted	value. My	hypothesis	
is	that	the	larger	the	frequency	of	the	vibration,	the	lesser	distance	of	the	
sound	at	its	highest	point

\section{method}
The	materials	needed	for	this	lab	was	a	tube,	an Apple's smart phone,	and	water.	We	
started	off	by	filling part of	the	tube	with	water.	Then	my teacher turn on his phone and hold it	over	the	top	of	the	tube.	Then we
listening	for	the	position	of	loudest	sound.	After	finding	the	antinode	we	measured	the	
distance	from	the	top	of	the	tube	to	the	water	surface	and	recorded	that	length.	This	
process	was	repeated	3	more	times	with		different	frequency.	

\section{result}
by using these formulas λ= 4*L and V= λ*f we came out with the data: \\
trial 1: \\
f= 512 Hz \\
L= 16,5 cm \\
trial 2: \\
f= 384 Hz \\
L= 22 cm \\
trial 3: \\
f= 320 Hz \\
L= 25 cm \\
trial 4: \\
f= 220 Hz \\
L= 38 cm \\
( the room temperture was 19C)\\
You	can	tell	that	when	the	period	increases,	the	wavelength	does	as	well.	The	only	
numbers	that	are	actual	are	the	frequency	from the phone.	The	rest	of	the	data	is	based	on	our	
hearing	throughout	the	entire	lab.	

\section{error}
there could have been many errors in my lab like:\\
- the source of f\\
- the resonant chamber\\
- the location of mose of wave is hard to determine.

\section{conclusion}
This	lab	taught	me	more	about	resonance.	I	learned	that	speed	of	sound	is	changed	
according	to	the	temperature	in	the	room	at	the	time.	

\end{document}
